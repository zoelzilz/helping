% Options for packages loaded elsewhere
\PassOptionsToPackage{unicode}{hyperref}
\PassOptionsToPackage{hyphens}{url}
%
\documentclass[
]{article}
\title{Steph Sierra Mosquitos}
\author{Zoe}
\date{6/1/2022}

\usepackage{amsmath,amssymb}
\usepackage{lmodern}
\usepackage{iftex}
\ifPDFTeX
  \usepackage[T1]{fontenc}
  \usepackage[utf8]{inputenc}
  \usepackage{textcomp} % provide euro and other symbols
\else % if luatex or xetex
  \usepackage{unicode-math}
  \defaultfontfeatures{Scale=MatchLowercase}
  \defaultfontfeatures[\rmfamily]{Ligatures=TeX,Scale=1}
\fi
% Use upquote if available, for straight quotes in verbatim environments
\IfFileExists{upquote.sty}{\usepackage{upquote}}{}
\IfFileExists{microtype.sty}{% use microtype if available
  \usepackage[]{microtype}
  \UseMicrotypeSet[protrusion]{basicmath} % disable protrusion for tt fonts
}{}
\makeatletter
\@ifundefined{KOMAClassName}{% if non-KOMA class
  \IfFileExists{parskip.sty}{%
    \usepackage{parskip}
  }{% else
    \setlength{\parindent}{0pt}
    \setlength{\parskip}{6pt plus 2pt minus 1pt}}
}{% if KOMA class
  \KOMAoptions{parskip=half}}
\makeatother
\usepackage{xcolor}
\IfFileExists{xurl.sty}{\usepackage{xurl}}{} % add URL line breaks if available
\IfFileExists{bookmark.sty}{\usepackage{bookmark}}{\usepackage{hyperref}}
\hypersetup{
  pdftitle={Steph Sierra Mosquitos},
  pdfauthor={Zoe},
  hidelinks,
  pdfcreator={LaTeX via pandoc}}
\urlstyle{same} % disable monospaced font for URLs
\usepackage[margin=1in]{geometry}
\usepackage{color}
\usepackage{fancyvrb}
\newcommand{\VerbBar}{|}
\newcommand{\VERB}{\Verb[commandchars=\\\{\}]}
\DefineVerbatimEnvironment{Highlighting}{Verbatim}{commandchars=\\\{\}}
% Add ',fontsize=\small' for more characters per line
\usepackage{framed}
\definecolor{shadecolor}{RGB}{248,248,248}
\newenvironment{Shaded}{\begin{snugshade}}{\end{snugshade}}
\newcommand{\AlertTok}[1]{\textcolor[rgb]{0.94,0.16,0.16}{#1}}
\newcommand{\AnnotationTok}[1]{\textcolor[rgb]{0.56,0.35,0.01}{\textbf{\textit{#1}}}}
\newcommand{\AttributeTok}[1]{\textcolor[rgb]{0.77,0.63,0.00}{#1}}
\newcommand{\BaseNTok}[1]{\textcolor[rgb]{0.00,0.00,0.81}{#1}}
\newcommand{\BuiltInTok}[1]{#1}
\newcommand{\CharTok}[1]{\textcolor[rgb]{0.31,0.60,0.02}{#1}}
\newcommand{\CommentTok}[1]{\textcolor[rgb]{0.56,0.35,0.01}{\textit{#1}}}
\newcommand{\CommentVarTok}[1]{\textcolor[rgb]{0.56,0.35,0.01}{\textbf{\textit{#1}}}}
\newcommand{\ConstantTok}[1]{\textcolor[rgb]{0.00,0.00,0.00}{#1}}
\newcommand{\ControlFlowTok}[1]{\textcolor[rgb]{0.13,0.29,0.53}{\textbf{#1}}}
\newcommand{\DataTypeTok}[1]{\textcolor[rgb]{0.13,0.29,0.53}{#1}}
\newcommand{\DecValTok}[1]{\textcolor[rgb]{0.00,0.00,0.81}{#1}}
\newcommand{\DocumentationTok}[1]{\textcolor[rgb]{0.56,0.35,0.01}{\textbf{\textit{#1}}}}
\newcommand{\ErrorTok}[1]{\textcolor[rgb]{0.64,0.00,0.00}{\textbf{#1}}}
\newcommand{\ExtensionTok}[1]{#1}
\newcommand{\FloatTok}[1]{\textcolor[rgb]{0.00,0.00,0.81}{#1}}
\newcommand{\FunctionTok}[1]{\textcolor[rgb]{0.00,0.00,0.00}{#1}}
\newcommand{\ImportTok}[1]{#1}
\newcommand{\InformationTok}[1]{\textcolor[rgb]{0.56,0.35,0.01}{\textbf{\textit{#1}}}}
\newcommand{\KeywordTok}[1]{\textcolor[rgb]{0.13,0.29,0.53}{\textbf{#1}}}
\newcommand{\NormalTok}[1]{#1}
\newcommand{\OperatorTok}[1]{\textcolor[rgb]{0.81,0.36,0.00}{\textbf{#1}}}
\newcommand{\OtherTok}[1]{\textcolor[rgb]{0.56,0.35,0.01}{#1}}
\newcommand{\PreprocessorTok}[1]{\textcolor[rgb]{0.56,0.35,0.01}{\textit{#1}}}
\newcommand{\RegionMarkerTok}[1]{#1}
\newcommand{\SpecialCharTok}[1]{\textcolor[rgb]{0.00,0.00,0.00}{#1}}
\newcommand{\SpecialStringTok}[1]{\textcolor[rgb]{0.31,0.60,0.02}{#1}}
\newcommand{\StringTok}[1]{\textcolor[rgb]{0.31,0.60,0.02}{#1}}
\newcommand{\VariableTok}[1]{\textcolor[rgb]{0.00,0.00,0.00}{#1}}
\newcommand{\VerbatimStringTok}[1]{\textcolor[rgb]{0.31,0.60,0.02}{#1}}
\newcommand{\WarningTok}[1]{\textcolor[rgb]{0.56,0.35,0.01}{\textbf{\textit{#1}}}}
\usepackage{graphicx}
\makeatletter
\def\maxwidth{\ifdim\Gin@nat@width>\linewidth\linewidth\else\Gin@nat@width\fi}
\def\maxheight{\ifdim\Gin@nat@height>\textheight\textheight\else\Gin@nat@height\fi}
\makeatother
% Scale images if necessary, so that they will not overflow the page
% margins by default, and it is still possible to overwrite the defaults
% using explicit options in \includegraphics[width, height, ...]{}
\setkeys{Gin}{width=\maxwidth,height=\maxheight,keepaspectratio}
% Set default figure placement to htbp
\makeatletter
\def\fps@figure{htbp}
\makeatother
\setlength{\emergencystretch}{3em} % prevent overfull lines
\providecommand{\tightlist}{%
  \setlength{\itemsep}{0pt}\setlength{\parskip}{0pt}}
\setcounter{secnumdepth}{-\maxdimen} % remove section numbering
\ifLuaTeX
  \usepackage{selnolig}  % disable illegal ligatures
\fi

\begin{document}
\maketitle

\emph{John's Data}

\begin{Shaded}
\begin{Highlighting}[]
\NormalTok{mosq }\OtherTok{\textless{}{-}} \FunctionTok{read\_csv}\NormalTok{(}\StringTok{"mosq2019\_sjc.csv"}\NormalTok{) }\SpecialCharTok{\%\textgreater{}\%} 
  \CommentTok{\#for analyses, only need fish (ind), elevation (ind), lake?, direction (ind) trap\_id (random), date (random), tot\_mosq(response var)}
  \FunctionTok{select}\NormalTok{(tot\_mosq, fish, elevation, lake, trap\_id, date, direction)}
\end{Highlighting}
\end{Shaded}

\begin{verbatim}
## Rows: 50 Columns: 13
\end{verbatim}

\begin{verbatim}
## -- Column specification --------------------------------------------------------
## Delimiter: ","
## chr  (9): lake, trap_num, direction, trap_id, date, lat, long, elevation, veg
## dbl  (2): fish, tot_mosq
## time (2): time_set, time_collected
\end{verbatim}

\begin{verbatim}
## 
## i Use `spec()` to retrieve the full column specification for this data.
## i Specify the column types or set `show_col_types = FALSE` to quiet this message.
\end{verbatim}

\begin{Shaded}
\begin{Highlighting}[]
\CommentTok{\# fish needs to be a factor:}
\NormalTok{mosq}\SpecialCharTok{$}\NormalTok{fish }\OtherTok{\textless{}{-}} \FunctionTok{factor}\NormalTok{(mosq}\SpecialCharTok{$}\NormalTok{fish)}
\NormalTok{mosq}\SpecialCharTok{$}\NormalTok{elevation }\OtherTok{\textless{}{-}} \FunctionTok{factor}\NormalTok{(mosq}\SpecialCharTok{$}\NormalTok{elevation, }\AttributeTok{levels =} \FunctionTok{c}\NormalTok{(}\StringTok{"low"}\NormalTok{, }\StringTok{"mid"}\NormalTok{, }\StringTok{"high"}\NormalTok{))}
\end{Highlighting}
\end{Shaded}

\hypertarget{lets-do-some-visualization-to-see-where-our-patternsvariation-is.-i-like-to-use-scatter-plots}{%
\subsection{let's do some visualization to see where our
patterns/variation is. I like to use scatter
plots}\label{lets-do-some-visualization-to-see-where-our-patternsvariation-is.-i-like-to-use-scatter-plots}}

steph's qs: - What happens to abundance by trap night? - Does it
systematically decrease? - Some effect of cardinal direction?

\begin{Shaded}
\begin{Highlighting}[]
\NormalTok{trap\_id\_effect }\OtherTok{\textless{}{-}} \FunctionTok{ggplot}\NormalTok{(mosq, }\FunctionTok{aes}\NormalTok{(}\AttributeTok{x =}\NormalTok{ elevation, }\AttributeTok{y =}\NormalTok{ tot\_mosq)) }\SpecialCharTok{+}
  \FunctionTok{geom\_point}\NormalTok{(}\FunctionTok{aes}\NormalTok{(}\AttributeTok{color =}\NormalTok{ trap\_id))}
\NormalTok{trap\_id\_effect}
\end{Highlighting}
\end{Shaded}

\includegraphics{sierra_mosquitoes_zz_files/figure-latex/data vis-1.pdf}

\begin{Shaded}
\begin{Highlighting}[]
\CommentTok{\# new mac is broken so to ggplot, have to copy graph name and put in console}

\CommentTok{\# no noticable effect of trap\_id, should prob code as random}

\NormalTok{date\_effect }\OtherTok{\textless{}{-}} \FunctionTok{ggplot}\NormalTok{(mosq, }\FunctionTok{aes}\NormalTok{(}\AttributeTok{x =}\NormalTok{ date, }\AttributeTok{y =}\NormalTok{ tot\_mosq)) }\SpecialCharTok{+}
  \FunctionTok{geom\_point}\NormalTok{()}
\NormalTok{date\_effect}
\end{Highlighting}
\end{Shaded}

\includegraphics{sierra_mosquitoes_zz_files/figure-latex/data vis-2.pdf}

\begin{Shaded}
\begin{Highlighting}[]
\CommentTok{\#maybe an increase in mosquitoes by date? might want to include an interaction of date and elev in model}

\NormalTok{fish\_effect }\OtherTok{\textless{}{-}} \FunctionTok{ggplot}\NormalTok{(mosq, }\FunctionTok{aes}\NormalTok{(}\AttributeTok{x =}\NormalTok{ fish, }\AttributeTok{y =}\NormalTok{ tot\_mosq)) }\SpecialCharTok{+}
  \FunctionTok{geom\_point}\NormalTok{(}\FunctionTok{aes}\NormalTok{(}\AttributeTok{color =}\NormalTok{ elevation))}
\CommentTok{\# more mosquitoes with fish}

\NormalTok{direction\_effect }\OtherTok{\textless{}{-}} \FunctionTok{ggplot}\NormalTok{(mosq, }\FunctionTok{aes}\NormalTok{(}\AttributeTok{x =}\NormalTok{ direction, }\AttributeTok{y =}\NormalTok{ tot\_mosq)) }\SpecialCharTok{+}
  \FunctionTok{geom\_point}\NormalTok{(}\FunctionTok{aes}\NormalTok{(}\AttributeTok{color =}\NormalTok{ elevation))}
\CommentTok{\# N and S facing traps look like they have more mosquitoes}
\end{Highlighting}
\end{Shaded}

\emph{Box/Whisker plots of the data} Tejon style box plots would be nice

What is the variation treatment (fish) by level (elevation)?

\begin{Shaded}
\begin{Highlighting}[]
\CommentTok{\# boxplot with differences between fish/no fish, facet wrapped by elevation (hi/med/lo)}
\NormalTok{fish\_elev }\OtherTok{\textless{}{-}} \FunctionTok{ggplot}\NormalTok{(mosq, }\FunctionTok{aes}\NormalTok{(}\AttributeTok{x =}\NormalTok{ fish, }\AttributeTok{y =}\NormalTok{ tot\_mosq)) }\SpecialCharTok{+}
  \FunctionTok{geom\_boxplot}\NormalTok{()}\SpecialCharTok{+}
  \FunctionTok{facet\_wrap}\NormalTok{(}\FunctionTok{vars}\NormalTok{(elevation))}
\NormalTok{fish\_elev}
\end{Highlighting}
\end{Shaded}

\includegraphics{sierra_mosquitoes_zz_files/figure-latex/boxplots-1.pdf}

\begin{Shaded}
\begin{Highlighting}[]
\CommentTok{\# no big diff at high, big diffs at low and med}
\end{Highlighting}
\end{Shaded}

Mixed effect model - Mos abundance = Fish status (Y/N) + Elevation
(High, Medium, Low) + \url{Fish:Elevation}

What's the replicate? Every night and/or every trap as if it was an
independent sample? (every night is how data is set currently) Pooling
them to a single trap? Or run with random effect of Trap ID? What is the
result of that? Is there something we can confidently say is
significant?

\end{document}
